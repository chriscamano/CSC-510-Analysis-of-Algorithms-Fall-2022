
\documentclass[12pt]{article}
\usepackage[pdftex]{graphicx}
\usepackage{amsmath,amssymb,amsthm}
\usepackage{hyperref}
\pagestyle{empty}
\author{Chris Camano: ccamano@sfsu.edu}
\title{MATH 510  Lecture 2 }
\date

\topmargin -0.6in
\headsep 0.40in
\oddsidemargin 0.0in
\textheight 9.0in
\textwidth 6.5in
\vfuzz2pt
\hfuzz2pt

%%%%Short cuts and formatting%%%%%%%%%%
\newcommand{\q}{\quad}
\newcommand{\tab}{\\\\}
\renewcommand{\labelenumi}{\alph{enumi})}
\newcommand{\sect}[1]{\section*{#1}}

%%%%%%Vector Spaces%%%%%%%%%%%%%%%%%%%
\newcommand{\R}{\mathbb{R}}
\newcommand{\C}{\mathbb{C}}
\newcommand{\F}{\mathbb{F}}
\newcommand{\Q}{\mathbb{Q}}
\newcommand{\rtwo}{\mathbb{R}^2}
\newcommand{\mxn}{{m\times n}}

%%%%%%Sets and common phrases%%%%%%%%%
\newcommand{\Axb}{\textbf{Ax=b} }
\newcommand{\Axz}{\textbf{Ax=0} }
\newcommand{\dim}{\text{dim}}
\newcommand{\lc}{linear combination }
\newcommand{\let}{\text{Let }}
\newcommand{\tf}{\therefore}
%%%%%%%%%Analysis%%%%%%%%%%%%%%%%%%%%%
\newcommand{\arr}{\rightarrow}
\newcommand{\xlim}{\lim_{x\rightarrow \infty}}
\newcommand{\Z}{\mathbb{Z}}
\newcommand{\N}{\mathbb{N}}
\newcommand{\ep}{\varepsilon}
\newcommand{\i}{\text{ if }}
\newcommand{\and}{\text{ and }}
%%%%%% Theorem formatting%%%%%%%%%%%
\newtheorem{thm}{Theorem}[section]
\newtheorem{cor}[thm]{Corollary}
\newtheorem{lem}[thm]{Lemma}
\newtheorem{prop}[thm]{Proposition}
\theoremstyle{definition}
\newtheorem{defn}[thm]{Definition}
\theoremstyle{remark}
\newtheorem{rem}[thm]{Remark}
\numberwithin{equation}{section}
\everymath={\displaystyle}


\begin{document}
\maketitle


\sect{Big O notation II}\\
As a refresher K is the point in which the function in question dominates another function for $n>K$.\\
Generally when solving problems of this nature it is a good place to start by identifying the constant C on the other side of the inequality\\
This can be done by selecting the next integer after the coefficient on the term on the left hand side with the highest degree. C can be decremented to a non integer value as long as that value is greater than the largest coefficient.
\[
f(x)=  a_kx^k+a_{k-1}x^{k-1}+\cdots+a_1x^1+a_0
\]
so long as $C>a_k$ the function will dominate for some K.\\\
f(x) is O(g(x)) $\iff$$\exists C,K\in \R^+:$
\[
  |f(n)|\leq C|g(n)|\quad \forall n\geq K
\]
Big O notation functions purely as an upper bound for lower bounds we consider Big Omega notation and if a function is bounded by some other function above and below we then turn our attention to big Theta. \\
\defn Change of base formula:
\[
  log_a(n)=\frac{log_x(n)}{log_x(a)}
\]
\prop Modifying the definition of Big O notation:
\[
f(x)=  a_kx^k+a_{k-1}x^{k-1}+\cdots+a_1x^1+a_0
\]
so long as $C>a_k$ the function will dominate for some K.\\\
f(x) is O(g(x)) $\iff$$\exists C,K\in \R^+:$
\[
  |f(n)|\leq C|g(n)|\quad \forall n\geq K
\]
\[
  \lim_{n\arr\infty}\frac{|f(n)|}{|g(n)|}\leq C\quad \forall n\geq K
\]
Example:
\[
  \lim_{n\arr\infty}\frac{|n^{100}|}{|e^n|}\leq C\quad \forall n\geq K
\]
\[
  \lim_{n\arr\infty}\frac{|100n^{99}|}{|e^n|}\leq C\quad \forall n\geq K
\]
\[
  \lim_{n\arr\infty}\frac{100!}{|e^n|}\leq C\quad \forall n\geq K
\]
\[
0\leq C\quad \forall n\geq K
\]\\\\
Given a polynomial function. First eliminate negative terms by expressing an inequality in which the negative terms have been removed.
\[
  f(x)\leq f(x)^+ ,\quad f(x) \in \mathbb{P}
\]
Next, adjust all of the polynomial terms to share the highest degree and sum the terms to compute a leading coefficient. In general the sum of the coefficients is a good choice of C for a big O classification. This is true for all values greater than one due to the behavior of exponentating rational numbers.
\\\\
constant$<$logathrithmic$<$polynomial$<$exponential\\
if(f(n)=O(g(n))) and h(n)$>$0 then\\ f(n)h(n)=O(g(n)h(n))\\
\defn Big O properties\\
$$O(f(n))+O(g(n))=O(f(n)+g(n))$$\\
$$O(f(n))O(g(n))=O(f(n)g(n))$$\\
$$O(f(n))+g(n))=O(max(f(n),g(n))$$\\
\prop Example:
Given a term such as
\[
  f(n)=20(n^2+n^2log_4n)(4n+3)+(17log_3n+19)(n^3+2)
\]
First take the fastest  function thats increasing
\[
  f(n)=O(n^2\log(n))O(n)+O(log(n))O(n^3)
\]
\[
  f(n)=O(n^3\log(n))+O(n^3log(n))
\]
\[
  f(n)=O(n^3\log(n))
\]
\defn Big Omega: \\
f(x) is $\Omega$(g(x)) $\iff$$\exists C,K\in \R^+:$
\[
  |f(n)|\geq C|g(n)|\quad \forall n\geq K
\]
\defn Big Theta
f(x) is $\Theta$(g(x)) $\iff$$\exists C,K\in \R^+:$
\[
C|g(n)|\leq  |f(n)|\leq C|g(n)|\quad \forall n\geq K
\]
\defn Little o:
\[
  |f(n)| < C|g(n)|\quad \forall n\geq K
\]
\defn Little $\omega$:
\[
  |f(n)| > C|g(n)|\quad \forall n\geq K
\]
\sect{Pseduocode and Analyzing Runtime}\\
Do not submit any code in this course, use Pseduocode with natural language semantics. \\
\defn Analyzing iterative algorithms. \\
Giving a big O estimate for the return value cout in the function: If loops are not nested sum them if they are multiply them
\end{document}
